\section{Related Works}
Statistical Model Checking technique was first proposed by R.Grosu \cite{grosu2005monte}. Some variations \cite{legay2010statistical} \cite{Younes2004Planning} \cite{younes2006statistical}\cite{jha2009bayesian} \cite{zuliani2013bayesian} \cite{herault2004} based on the basic SMC have been proposed in the past few years. Some related work are summarized as follows:

\textbf{Basic SMC.}
SMC refers to a series of simulation-based techniques that can be used to answer two questions: (1)Qualitative: Is the probability of model (s) satisfying property ($\phi$) greater than or equal to a certain threshold? and (2)Quantitative: What is the probability of model (s) satisfying property ($\phi$)? For qualitative SMC, Kim et al. \cite{kim2012statistical} have given an empirical evaluation. BHT and SPRT are more effective than SSP. BHT generates more traces when checking the property whose estimation probability is close to its real probability, thus SPRT is faster than BHT in this situation, otherwise BHT is obviously more efficient than SPRT. For quantitative SMC, Zuliani et al. \cite{zuliani2013bayesian} have compared the number of traces analyzed by APMC and BIE, and they have concluded that BIE excels remarkably in performance. Our approach focuses on the performance of BIE algorithm.

\textbf{SMC with abstraction and learning.}
BIE algorithm needs more traces when checking the property whose probability is close to 0.5, while the number of traces is drastically reduced when the probability approaches to 0 or 1 \cite{zuliani2013bayesian}. In our recent work \cite{jiangkaiqiang2016}, we have partitioned the original probability space ($\Omega$) into many sub-spaces $\Omega_1$,... ,$\Omega_m$, and evaluated the probability of each sub-space in parallel. Therefore, the trace number for evaluating the original probability will be decreased and depends on the maximum number of traces for evaluating sub-spaces theoretically. We find that the number of traces is effectively reduced while ensuring the accuracy of the probability within an acceptable error bound.

\textbf{Distributed SMC.}
As observed in \cite{younes2005ymer}, SMC algorithms can be distributed with master/slave architecture where multiple slave processes are used to generate traces. When working with an estimation algorithm, the number of traces for verifying the property is known in advance and can be equally distributed between the slaves. When working with the sequential algorithms, the situation gets more complicated, so we need to avoid introducing bias when collecting the traces generated by the slave processes. To solve this problem, H. L. S. Younes proposed a method in \cite{Younes2004Planning} where the bias is avoided by committing ,$\alpha$ $priori$, to the order in which observations will be taken into account. Peter Bulychev et al. generalized the above method with batches and buffer \cite{Bulychev2012Checking}. Batches aggregate the outcomes for reducing communication and the buffer is used to improve concurrency since the nodes are more loosely synchronized. They also implemented the distributed Hypothesis testing algorithm without introducing bias. The algorithm effectively reduce the time consumption for generating a single trace. Our work is different from the existing work, we use abstraction and learning technique (AL-SMC) to reduce the number of simulation traces, and adopt distributed technology with AL-SMC to reduce both the number of traces and time consumption for generating a single traces.   