\section{Introduction}

Statistical Model Checking techniques (SMC) \cite{Younes2004Planning,Sen2004Statistical,herault2004} can be seen as a trade-off between testing and formal verification. Recently, SMC has been an alternative to standard model-checking in order to avoid the state-space explosion problem, especially for verifying Cyber-physical Systems (CPSs) \cite{Yoo2016Challenges}. The core idea of SMC is to decide whether the stochastic model satisfies a given property or to evaluate its probability of satisfaction by combining statistical techniques with Monte-Carlo simulation on model traces. Nowadays, SMC is getting increasing industrial attentions and there are many model checkers which supports SMC techniques to analyze the stochastic model more effectively (e.g., Uppaal-SMC \cite{Bulychev2012UPPAAL}, Prism \cite{Kwiatkowska2002PRISM}).

CPS focus on the coupling of cyber part viewed as distributed computation units and physical part covering the environment affecting the running of the system. The modeling of stochastic behaviors for CPS might be highly cumbersome \cite{basu2010statistical} and the analysis of these models demands extremely high confidence \cite{du2015smc4rare}. SMC still encounters the performance bottleneck for verifying CPS. There are two factors having a direct influence on the performance of SMC, one is the number of simulation traces, the other is the length of a single trace.  In this paper, we focus on these two factors to improve the efficiency of SMC. Figure \ref{tech-map} is the technology roadmap of our approach. A statistical model checker contains three components: \emph{simulator, SMC algorithm and model checker}. The simulator generates simulation traces which are feed to the model checker.  With model checking technology, it verifies whether the trace satisfies the property, and returns observations ( boolean or numerical values). The SMC algorithm collects observations obtained from a model checker and computes the probability with statistical testing. To reduce the number of simulation traces, we have proposed abstraction and learning teachique with SMC algorithm called AL-SMC \cite{jiangkaiqiang2016}. With many experiments, we find that AL-SMC effectively reduces the number of traces, and improves the performance of SMC. As observed in \cite{younes2005ymer}, SMC can be distributed based on master/slave architecture where several slave computers are used to generate simulation traces. Inspired by their work, we apply distributed technology on \emph{simulator} to reduce the time consumption for generating a single trace. Further, we propose the distributed Bayesian Interval Estimation (BIE) algorithm \cite{zuliani2013bayesian}. In this paper, we combine distributed technology with AL-SMC technique, and propose distributed SMC with abstraction and learning (DAL-SMC) technique.  DAL-SMC reduces both the number of simulation traces and the time consumption for generating a single trace.

\begin{figure}[htbp]
	{
	\centering	
	\includegraphics[width=3.0in,height=1.8in]{fig/paper-framework.png}
\caption{Technology roadmap.}\label{tech-map}	
	}
	%\vspace{0.10in}
	
\end{figure}

\textbf{The main contributions of this paper include:} 
\begin{itemize}
\item
We propose a novel verification framework, which applies distributed technology to improve the efficiency of SMC.
\item
We propose the distributed BIE and DAL-SMC algorithms based on the framework. The parameter optimization method is introduced to reduce the statistical error of DAL-SMC. This work is an extension of our previous work \cite{jiangkaiqiang2016}.
\item
The distributed BIE and DAL-SMC algorithms are implemented in our ModanaOnline platform  \cite{Cheng2015Modana} (http
s://github.com/ECNU-MODANA/Modana-Online) to support the automatic process. 
\item
Several experiments are conducted to demonstrate that our approach effectively reduces the time consumption within an acceptable error bound.
\end{itemize}

The remainder of this paper is organized as follows. In Section 2, we present the framework of our approach, and several core algorithms of distributed BIE and DAL-SMC in details. Parameter optimization method of DAL-SMC is also presented. Section 3 provides the algorithm analysis of DAL-SMC. Section 4 presents the implementation of our approach in ModanaOnline platform. Several experiments are conducted with three benchmarks. The experimental results show that our DAL-SMC are efficient and feasible. The related work is discussed in Section 5. Finally, conclusions and directions of future research are presented in Section 6. 