\documentclass[conference]{IEEEtran}
\IEEEoverridecommandlockouts
% The preceding line is only needed to identify funding in the first footnote. If that is unneeded, please comment it out.
\usepackage{cite}
\usepackage{amsmath,amssymb,amsfonts}
\usepackage{algorithmic}
\usepackage{graphicx}
\usepackage{textcomp}
\def\BibTeX{{\rm B\kern-.05em{\sc i\kern-.025em b}\kern-.08em
    T\kern-.1667em\lower.7ex\hbox{E}\kern-.125emX}}
\begin{document}

\title{DSL4CS: Co-simulation Language for Heterogeneous CPS based on Gemoc*\\
%{\footnotesize \textsuperscript{*}Note: Sub-titles are not captured in Xplore and
%should not be used}
\thanks{Identify applicable funding agency here. If none, delete this.}
}

%\author{\IEEEauthorblockN{1\textsuperscript{st} Given Name Surname}
%\IEEEauthorblockA{\textit{dept. name of organization (of Aff.)} \\
%\textit{name of organization (of Aff.)}\\
%City, Country \\
%email address}
%\and
%\IEEEauthorblockN{2\textsuperscript{nd} Given Name Surname}
%\IEEEauthorblockA{\textit{dept. name of organization (of Aff.)} \\
%\textit{name of organization (of Aff.)}\\
%City, Country \\
%email address}
%\and
%\IEEEauthorblockN{3\textsuperscript{rd} Given Name Surname}
%\IEEEauthorblockA{\textit{dept. name of organization (of Aff.)} \\
%\textit{name of organization (of Aff.)}\\
%City, Country \\
%email address}
%\and
%\IEEEauthorblockN{4\textsuperscript{th} Given Name Surname}
%\IEEEauthorblockA{\textit{dept. name of organization (of Aff.)} \\
%\textit{name of organization (of Aff.)}\\
%City, Country \\
%email address}
%\and
%\IEEEauthorblockN{5\textsuperscript{th} Given Name Surname}
%\IEEEauthorblockA{\textit{dept. name of organization (of Aff.)} \\
%\textit{name of organization (of Aff.)}\\
%City, Country \\
%email address}
%\and
%\IEEEauthorblockN{6\textsuperscript{th} Given Name Surname}
%\IEEEauthorblockA{\textit{dept. name of organization (of Aff.)} \\
%\textit{name of organization (of Aff.)}\\
%City, Country \\
%email address}
%}

\maketitle

\begin{abstract}
Domain specific language is ...
co-simulation:
We propose a co-simulation language, introduce the metamodel, concrete syntax and semantics.
We implement it based on Gemoc framework. 

\end{abstract}

\begin{IEEEkeywords}
co-simulation,DSL, metamodel, Gemoc, CPS
\end{IEEEkeywords}
\section{Introduction}

\textit{Cyber-physical systems} (CPS) are integration of computation with physical processes whose behavior is defined by both computational and physical parts of the system \cite{Zanero17}. Embedded computers and networks monitor and control the physical processes, usually with feedback loops where physical processes affect computations and vice versa. The heterogeneity is one of the main characteristics of CPS. The components of CPS are of various types, requiring interfacing and interoperability across multiple platforms and different models of computation. Verifying coordination of heterogeneous CPS is a challenging problem. The coordination between heterogeneous components of CPS could be implemented with Functional Mock-up Interface (FMI) based co-simulation technology. The FMI standard was first developed in the MODELISAR project started in 2008 and supported by a large number of software companies and research centers \cite{ClauMODELISAR}. FMI supports simulation of complex systems composed of heterogeneous components, by coupling different models with their own solvers in a co-simulation environment.

In this paper, we focus on verifying the coordination of CPS which implemented with FMI 2.0 \cite{Cremona2006Automatic} based co-simulation. The key point is Master Algorithm (MA) \cite{AckerDVM15} and connector configuration between Functional Mock-up Units (FMUs) \cite{Tripakis15}, which specifies the orchestration and the exchange of data among FMUs during the whole coordination process. To ensure the correctness of coordination, we need to verify MA and connector configuration with model checking. However, MA is not a part of FMI standard. This implies that the user or tool vendor needs to develop a sophisticated orchestration algorithm for the problem at hand. 
%Is the MA deadlock free?
%Dose the MA satisfy the reachability? To solve these problem, we can verify the MAs with model checking technology.
There are three versions of MA \cite{BromanBGLMTW13}: fixed step algorithm, rollback algorithm and predictable step size algorithm. Rollback and predictable step size algorithms are based on the extension of FMI 2.0, which supports the rollback and a predict function. P.G. Larsen et al. \cite{Larsen2016Integrated} formally analysed the fixed step and rollback algorithms with the FDR3 refinement checker. However, there still lacks effective approach to verify the whole FMI-based coordination process. Based on our previous work \cite{LiuJWCD16}, we found that the simulation process of coordination is time-intensive. Therefore, it is reasonable to formalize the coordination with Timed Automaton (TA) \cite{BehrmannDLHPYH06}, which is the classic formalism for modeling real-time system. 
In this paper, we propose to model the MA with TA and verify the correctness of MA. Furthermore, we also attempt to encode the components of CPS with TA and verify the architecture of whole system with the model checker UPPAAL \cite{BehrmannDLHPYH06}. 
%To achieve our goal, we propose a novel approach to model check the coordination of CPS with TA.

\textbf{In summary, our main contributions are as follows}:
\begin{itemize}
\item
We propose a novel approach to verify the coordination of CPS with model checking. To bridge the gap between FMU and the model checker, we propose some encoding rules to encode FMU as TA.
\item
We model and verify three various MAs to ensure the correctness of the coordination. With the help of UPPAAL, we analyse the reachability, livelock and deadlock of three versions of MA.
%\item
%We present a novel approach to model check several properties of the co-simulation based on timed automata. With the help of model checker, the property such as livelock, deadlock and reachability of the co-simulation are verified. 
\item
The prototype for model checking coordination of CPS is under developing, which is integrated in our Modana platform \cite{Cheng2015Modana}. We have implemented the \textit{SysML modeling environment} and the \textit{co-simulator} to simulate CPS in Modana (https://github.com/ECNU-MODANA/AL-Modana.git) \cite{Fritzson1998Modelica}.
\end{itemize}
The main novelty of our work, compared with the previous work, is that we propose to verify coordination of CPS with TA-based model checking. As far as we know, there is few existing approaches supporting TA-based model checking for the coordination of CPS.

The remainder of this paper is organized as follows. In Section~\ref{sec:fmi}, we briefly review the technical background including FMI, FMU and TA. Then, we present the technical road map of our approach and discuss how to encode FMU as TA with the help of their semantic mapping rules in Section~\ref{sec:encoding}. In Section~\ref{sec:ma}, we model three versions of MA with TA and verify their properties such as the livelock and deadlock. Section~\ref{sec:sysml} presents a case study to demonstrate the feasibility of our approach. We model the architecture of water tank system with SysML Block Definition Diagram (BDD) \cite{SemerathBHSV17}, and then obtain the FMU component of each block and the connection between FMUs. We encode the FMUs of water tank with TA and verify the correctness of the coordination between components of water tank system with UPPAAL. Finally, we position our work with respect to related work before concluding and discussing possible future extensions.




















%The syntax of TA is as follows:
%\par
%\textbf{Timed Automata}
%A timed automata over a finite set of clocks $C$ and a finite set of actions $Sigma$ is a quintuple $\textit{H}=(L,l_{0},\Sigma,E,I)$, where
%\begin{itemize}
%\item
%$L$ is a finite set of locations,
%\item
%$l_{0}\in L$ is the initial location,
%\item
%$\Sigma$ is a finite set of actions, and $\Sigma=\Sigma_{i}+\Sigma_{o}$, where $\Sigma_{i}$ is the set of input actions, $\Sigma_{o}$ is the set of output actions,
%\item
%$E$ is a finite set of transactions, where $E\subseteq L \times \mathcal{B(C)}\times \Sigma \times 2^C\times L$
%\item
%$I:L\rightarrow \mathcal{B(C)})$ assigns invariants to locations.
%\end{itemize}
\section{Co-simulation Language for Heterogeneous CPS (DSL4CPS)}
The design of complex systems often relies on several Domain Specific Modeling Languages (DSMLs) that may pertain to different theoretical domains with different expected expressiveness and properties. As a result, several models conforming to different DSMLs are developed and the specification of the
overall system becomes heterogeneous.

Meta-models, as defined here, provide: A language representation with strict notation and grammar. This is used to represent and store the meta-model and to share meta-model information between different tools and people. 
Model abstraction, where each external model is represented individually in a uniform and simulation tool independent way. A generic and uniform way to connect various simulation tools. Platform independent models. No operating system, network, or other co-simulation platform dependent parameters are stored in the meta-model. A single meta-model simulator application can start the simulation tools and control the co-simulation based on the meta-model. 
\subsection{Meta Model for DSL4CPS}

\subsection{Concrete Syntax for DSL4CPS}
rules, director, coordinate pattern
\subsection{Semantics for DSL4CPS}
extend Gemoc to support DSL4CS
\section{Extend Gemoc to Support DSL4CPS}
\subsection{Design Meta Model Using EMF}
\subsection{Generate Concrete Syntax Based on Xtext}
\subsection{Define Semantics with CCSL}
\section{Implementation and Case Study}
\subsection{Tool Implementation based on Gemoc }
\subsection{Case Study}

\section{Related Works}
Statistical Model Checking technique was first proposed by R.Grosu \cite{grosu2005monte}. Some variations \cite{legay2010statistical} \cite{Younes2004Planning} \cite{younes2006statistical}\cite{jha2009bayesian} \cite{zuliani2013bayesian} \cite{herault2004} based on the basic SMC have been proposed in the past few years. Some related work are summarized as follows:

\textbf{Basic SMC.}
SMC refers to a series of simulation-based techniques that can be used to answer two questions: (1)Qualitative: Is the probability of model \emph{s} satisfying property $\phi$ greater than or equal to a certain threshold? and (2)Quantitative: What is the probability of model \emph{s} satisfying property $\phi$? For qualitative SMC, Kim.G.larsn et al. \cite{kim2012statistical} have given an empirical evaluation. BHT and SPRT are more effective than SSP. BHT generates more traces when checking the property whose estimation probability is close to its real probability, so SPRT is faster than BHT for this situation. For other situation, BHT is obviously more efficient than SPRT. For quantitative SMC, Zuliani et al. \cite{zuliani2013bayesian} have compared the number of traces analyzed by APMC and BIE, and they have concluded that BIE excels remarkably in performance. Our approach focuses on the performance of BIE algorithm.

\textbf{SMC with abstraction and learning.}
BIE algorithm needs more traces when checking the property whose probability is close to 0.5, while the number of traces is drastically reduced when the probability approaches to 0 or 1 \cite{zuliani2013bayesian}. In our recent work \cite{jiangkaiqiang2016}, we have partitioned the original probability space $\Omega$ into many sub-spaces $\Omega_1$,... ,$\Omega_m$, and evaluated the probability of each sub-space in parallel. Therefore, the trace number for evaluating the original probability will be decreased and depends on the maximum number of traces for evaluating sub-spaces theoretically. We find that the number of traces is effectively reduced while ensuring the accuracy of the probability within an acceptable error bound.

\textbf{Distributed SMC.}
As observed in \cite{younes2005ymer}, SMC algorithms can be distributed with master/slave architecture where multiple slave processes are used to generate traces. When working with an estimation algorithm, the number of traces for verifying the property is known in advance and can be equally distributed between the slaves. When working with the sequential algorithms, the situation gets more complicated, so we need to avoid introducing bias when collecting the traces generated by the slave processes. To solve this problem, H. L. S. Younes proposed a method in \cite{Younes2004Planning} where the bias is avoided by committing ,$\alpha$ $priori$, to the order in which observations will be taken into account. Peter Bulychev et al. generalized the above method with batches and buffer \cite{Bulychev2012Checking}. Batches aggregate the outcomes for reducing communication and the buffer is used to improve concurrency since the nodes are more loosely synchronized. They also implemented the distributed Hypothesis testing algorithm without introducing bias. The algorithm effectively reduce the time consumption for generating a single trace. Our work is different from the existing work, we use abstraction and learning technique (AL-SMC) to reduce the number of simulation traces, and adopt distributed technology with AL-SMC to reduce both the number of traces and time consumption for generating a single traces.   
\section{Conclusion and Future Work}
\label{sec:conclusion&ack}
This paper has presented our novel approach to check the FMI co-simulation , which facilitates the formal analysis of CPSs. This involves model checking the reachability, livelock and deadlock of three various master algorithms. Besides, the correctness and  relevant system properties of the architecture are also analysed. To achieve the goal, we encode the FMU and master algorithms with timed automata. Then the properties of the co-simulation are verified with UPPAAL. We evaluate this approach using the example water tank. The results show that our approach is feasible and useful.

An interesting direction of future work is that we attempt to analyse and compare the performance of various master algorithms in the future. Besides, more complex case studies will be conducted to check the scalability of proposed approach. The tool support for our approach should be improved further.
\section*{Acknowledgement}
This work was supported by NSFC (Grant No.61472140, 61202104) and NSF of Shanghai (Grant No. 14ZR1412500).








\section*{References}


\end{document}
