\section{Co-simulation Language for Heterogeneous CPS (DSL4CPS)}
The design of complex systems often relies on several Domain Specific Modeling Languages (DSMLs) that may pertain to different theoretical domains with different expected expressiveness and properties. As a result, several models conforming to different DSMLs are developed and the specification of the
overall system becomes heterogeneous.

Meta-models, as defined here, provide: A language representation with strict notation and grammar. This is used to represent and store the meta-model and to share meta-model information between different tools and people. 
Model abstraction, where each external model is represented individually in a uniform and simulation tool independent way. A generic and uniform way to connect various simulation tools. Platform independent models. No operating system, network, or other co-simulation platform dependent parameters are stored in the meta-model. A single meta-model simulator application can start the simulation tools and control the co-simulation based on the meta-model. 
\subsection{Meta Model for DSL4CPS}

\subsection{Concrete Syntax for DSL4CPS}
rules, director, coordinate pattern
\subsection{Semantics for DSL4CPS}
extend Gemoc to support DSL4CS