 
%%%%%%%%%%%%%%%%%%%%%%% file typeinst.tex %%%%%%%%%%%%%%%%%%%%%%%%%
%
% This is the LaTeX source for the instructions to authors using
% the LaTeX document class 'llncs.cls' for  contributions to
% the Lecture Notes in Computer Sciences series.
% http://www.springer.com/lncs       Springer Heidelberg 2006/05/04
%
% It may be used as a template for your own input - copy it
% to a new file with a new name and use it as the basis
% for your article.
%
% NB: the document class 'llncs' has its own and detailed documentation, see
% ftp://ftp.springer.de/data/pubftp/pub/tex/latex/llncs/latex2e/llncsdoc.pdf
%
%%%%%%%%%%%%%%%%%%%%%%%%%%%%%%%%%%%%%%%%%%%%%%%%%%%%%%%%%%%%%%%%%%%


\documentclass[runningheads,a4paper]{llncs}

\usepackage{amssymb}
\setcounter{tocdepth}{3}
\usepackage{graphicx}
\usepackage{booktabs} % For formal tables
\usepackage[ruled]{algorithm2e} % For algorithms
\usepackage{multirow}
\usepackage{subfigure}
\usepackage{geometry}
\usepackage{amsmath}
\usepackage{amssymb}
\geometry{a4paper,left=3.0cm,right=3.0cm
,top=3cm,bottom=3cm}
\usepackage{url}
\urldef{\mailsa}\path|{dhdu}@sei.ecnu.edu.cn|    
\newcommand{\keywords}[1]{\par\addvspace\baselineskip
\noindent\keywordname\enspace\ignorespaces#1}

\begin{document}


% first the title is needed
\title{Model Checking FMI Co-simulation Using Timed Automata
\thanks{This work was supported by NSFC (Grant No.61472140, 61202104) and NSF of Shanghai (Grant No. 14ZR1412500).}}

% a short form should be given in case it is too long for the running head
%\titlerunning{Lecture Notes in Computer Science: Authors' Instructions}

% the name(s) of the author(s) follow(s) next
%
% NB: Chinese authors should write their first names(s) in front of
% their surnames. This ensures that the names appear correctly in
% the running heads and the author index.
%
\author{Kaiqiang Jiang
\and Chunlin Guan\and Jiahui Wang\and Dehui Du*}
%
%\authorrunning{Lecture Notes in Computer Science: Authors' Instructions}
% (feature abused for this document to repeat the title also on left hand pages)

% the affiliations are given next; don't give your e-mail address
% unless you accept that it will be published
\institute{Shanghai Key Laboratory of Trustworthy Computing, \\School of Computer Science and Software Engineering, East China Normal University, Shanghai, China\\
\mailsa}

%
% NB: a more complex sample for affiliations and the mapping to the
% corresponding authors can be found in the file "llncs.dem"
% (search for the string "\mainmatter" where a contribution starts).
% "llncs.dem" accompanies the document class "llncs.cls".
%

%\toctitle{Lecture Notes in Computer Science}
%\tocauthor{Authors' Instructions}
\maketitle


\begin{abstract}
Cyber-physical Systems(CPSs) focus on the coupling of cyber part viewed as distributed computation units and physical part covering the environment affecting the running of the system. CPSs are often treated modularly to tackle both complexity and heterogeneity. The verification of CPSs may be done modularly by Functional Mock-up Interface (FMI) co-simulation. However, the master algorithm for co-simulation may be livelock or deadlock. The architectural modelling of CPSs may introduce an algebraic loop which is a feedback loop resulting in instantaneous cyclic dependencies. To solve these problems, we propose a novel approach for model checking several properties of FMI co-simulation such as deadlock, liveness, reachability. 
We model the architecture of CPSs with SysML block diagrams, which captures the dependence of Functional Mock-up units (FMUs) and the orchestration of the master algorithm. Next, we encode FMU components and three various master algorithms with timed automata separately. Finally, we verify the correctness of the co-simulation and the absence of algebraic loops in the architecture with UPPAAL. To illustrate the feasibility of our approach, the case study water tank is presented. The results show that our approach helps model checking FMI co-simulation.
\keywords{Co-simulation, Master algorithm, Functional Mock-up Interface, Timed automata, Model checking}
\end{abstract}


\section{Introduction}
\textit{Cyber-physical systems} (CPSs) are integration of computation with physical processes whose behavior is defined by both computational and physical parts of the system \cite{Zanero17}. Embedded computers and networks monitor and control the physical processes, usually with feedback loops where physical processes affect computations and vice versa. The heterogeneity is one of the main characteristics of CPSs. The components of CPSs are of various types, requiring interfacing and interoperability across multiple platforms and different models of computation. Verifying  heterogeneous CPSs requires the use of heterogeneous simulation environments. One emerging industry standard is the Functional Mock-up Interface (FMI) \cite{Blochwitz2011The}\cite{BromanBGLMTW13}. It is a standard to support simulation of complex systems composed of heterogeneous components, by coupling the different models with their own solver in a co-simulation environment.

The FMI standard was first developed in the MODELISAR project started in 2008 and supported by a large number of software companies and research centers \cite{ClauMODELISAR}. FMI offers the means for model based development of systems and is particularly appropriate way to develop complex CPSs.  However, there  are 
\section{Encoding FMUs by timed automata}
\label{sec:fmi}
We give the syntax and semantics of FMU and timed automata. In order to verify the  execution of FMUs. We propose to encode FMUs by timed automata. In section \ref{sec:sysml}, we verify the network of timed automata with UPPAAL.
\subsection{FMU}
FMU is the model component which implements the methods defined in the FMI API \cite{Tripakis15}. Here, we present the syntax and semantics of FMU. The aim is to encode FMU into timed automata based on their semantics. 
\begin{definition}
\textbf{FMU syntax}
We recall the definition of FMU. An FMU is a tuple $F=(S,U,Y,D,s_{0},set,get,doStep)$, where:
\end{definition}
\begin{itemize}
\item
$S$ denotes the set of states of $F$. 
\item
$U$ denotes the set of input port variables of $F$. Note that an element $u \in U$ is a variable, not a value, which ranges over a set of values $\mathbb{V}$. 
\item
$Y$ denotes the set of output port variables of $F$. Each $y \in Y$ ranges over the same set of values $\mathbb{V}$.
\item
$D \subseteq U \times Y$ denotes a set of input-output dependencies. $(u,y) \in D $ means that the output y is directly dependent on the value of u. The $I/O$ dependency information is used to ensure that a network of FMUs does not contain cyclic dependencies, and also to identify the order in which all variables are computed during a step.
\item
$s_{0} \in S$ denotes the initial state of $F$.
\item
$set : S \times U \times \mathbb{V} \rightarrow S$ denotes the function that sets the value of an input variable. Given current state $s \in S$, input variable $u \in U$, and value $v \in \mathbb{V}$, it returns the new state obtained by setting $u$ to $v$.
\item
$get : S \times Y \rightarrow \mathbb{V}$ denotes the function that returns the value of an output variable. Given state $s \in S$ and output variable $y \in Y$, $get(s,y)$ returns the value of $y$ in $s$.
\item
$doStep : S \times \mathbb{R}_{\geqslant{0}} \rightarrow S \times \mathbb{R}_{\geqslant{0}}$ denotes the function that implements one simulation step. Given current state $s$, and a non-negative real value $h \in \mathbb{R}_{\geqslant{0}}$, $doStep(s,h)$ returns a pair $(s^{\prime},h^{\prime})$ such that:
\\
When $h^{\prime} = h$, it indicates that $F$ accepts the time step $h$ and reaches the new state $s^{\prime}$;
\\
When $0 \leqslant h^{\prime} < h$, this means that $F$ rejects the time step $h$, but making partial progress up to $h^{\prime}$, and reach the new location $s^{\prime}$.
\end{itemize}
\begin{definition}
\textbf{FMU semantics}
Given the FMU $F=(S,U,Y,D,s_{0},set,get,doStep)$,
\end{definition} 
The behavior of $F$ depends on the functions $doStep$, which is a function of a timed input sequence (TIS). A TIS is an infinite sequence 
$v_{0}h_{1}v_{1}h_{2}v_{2}h_{3}...$
of alternating input assignments $v_{i}$, and time delays $h_{i}$. An input assignment is the value of function $v : U \rightarrow \mathbb{V}$. That is, $v$ assigns a value to every input variable in $U$.
A TIS denotes a running of $FMU$, which is an infinite sequence of quadruples $(t,s,v,v^{\prime})$, where $t \in \mathbb{R}_{\geqslant{0}}$ is a time instant, $s \in S$ is a state of $F$, $v$ is an input assignment, and $v^{\prime} : Y \rightarrow \mathbb{V}$ is an output assignment
 
TIS := $(t_{0},s_{0},v_{0},v_{0}^{\prime})(t_{1},s_{1},v_{1},v_{1}^{\prime})(t_{2},s_{2},v_{2},v_{2}^{\prime})...$ is
defined as follows:
\begin{itemize}
\item
$t_{0} = 0$ and $s_{0}$ is the initial state of $F$.
\item
For each $i \geqslant 1$, $t_{i} = t_{0} + \sum_{k = 1}^i h_{k}$
\item
Given the current state $s_{i}$, the function $set$ is used to set all input variables to the values specified by $v$. Then $F$ reaches a new state $s_{i}^{\prime}$. The function $get$ is used to get the values of all output variables $v_{i}^{\prime}$.
\item 
We assume that $doStep(s_{i}, h_{i+1}) = (s_{i+1},h_{i+1})$ based on the assumption that every $h_{i}$ is accepted by $F$, $F$ will reach the next state $s_{i+1}$.
\end{itemize}
\subsection{Timed Automata}
Timed automata (TA) \cite{BehrmannDLHPYH06} is a theory to model the behavior of real-time systems. Its definition provides a powerful way to annotate state-transition graphs with many real-valued clocks. In this section, we introduce the syntax and semantics of timed automata. In section \ref{sec:sysml}, we will encode the FMUs of our case study with the network of TA, so that we can use the model checker UPPAAL to analyse models.
\begin{definition}
\textbf{Timed automata syntax}
A timed automaton is a tuple $\textit{A}=(L,X,l_{0},E_{i},E_{o},I)$, where:
\end{definition}
\begin{itemize}
\item
L is a finite set of locations;
\item
X is a finite set of clocks;
\item
$l_{0} \in  L$ is the initial state;
\item
The set of guards $G(x)$ is defined by the grammar $g := x \bowtie c \mid g \land g$, where $x \in X$, $c \in \mathbb{N}$ and $\bowtie~\in \{<,\leqslant,\geqslant,>\}$. $E \subseteq L \times G(X) \times 2^X \times L$ is a set of edges labelled by guards and a set of clocks to be reset;
\item
$E_{i}$ is a set of input events.
\item
$E_{o}$ is a set of output events.
\item
$I : L \rightarrow G(X)$ assigns invariants to clocks.
\end{itemize}
A clock valuation is a function $v : X \rightarrow \mathbb{R}_{\geqslant{0}}$. If $\delta \in \mathbb{R}_{\geqslant{0}}$, then $v + \delta$ denotes the valuation such that for each clock $x \in X$, $(v + \delta)(x) = v(x) + \delta$. If $Y \subseteq X$, then $v[Y := 0]$ denotes the valuation such that for each clock $x \in X~Y$, $v[Y := 0](x) = v(x)$ and for each clock $x \in Y$, $v[Y := 0](x) = 0$. The satisfaction relation $v \models g$ for $g \in G(x)$ is defined in the natural way.
\begin{definition}
\textbf{Timed automata semantics} 
The semantics of a timed automaton $\textit{A} = (L,X,l_{0},E,E_{i},E_{o},I)$ is defined by a transition system $L_{\textit{A}} = (L,l_{0},\rightarrow)$, 
\end{definition}
where $L = L \times \mathbb{R}_{\geqslant{0}}^X$ is the set of locations, $l_{0} = (l_{0},v_{0})$ is the initial location, $v_{0}(x) = 0$ for all $x \in X$, and $\rightarrow \subseteq L \times L$ is the set of transitions defined by :
\begin{itemize}
\item
$(l,v) \xrightarrow{\in(\delta)} (l,v+\delta)$ if $\forall0 \leqslant \delta^{\prime} \leqslant \delta : (v + \delta^{\prime}) \models I(s)$;
\item
$(l,v) \rightarrow (l^{\prime},v[Y := 0])$ if there exists $(l,g,Y,l^{\prime}) \in E$ such that $v \models g$ and $v[Y := 0] \models I(l^{\prime})$.
\end{itemize}
The reachability problem for an automaton $A$ and a location $l$ is to decide whether there is a state $(l,v)$ reachable from $(l_{0},v_{0})$ in the transition system $L_{A}$. As usual, for verification purposes, we define a symbolic semantics for timed automata. For universality, the definition uses arbitrary sets of clock valuations.

Consider a location $l$ such that for any $t \in X$, for fixed constant $x \in X$, clock valuation $x + t \in X$. A possible execution fragment starting from this location is

$(l,t) \xrightarrow{x_{1}} (l,t+x_{1}) \xrightarrow{x_{2}} (l,t+x_{1}+x_{2}) \xrightarrow{x_{3}} (l,t+x_{1}+x_{2}+x_{3}) \xrightarrow{x_{4}}...$

where $x_{i} > 0$ and the infinite sequence $x_{1} + x_{2} + . . .$ converges toward $x$. 
\subsection{Encoding FMUs into timed automata}
As we can see, there is a semantic gap between FMU and TA. The former focus on the execution sequence of FMU, which specifies the state change process with time passing. Essentially, the execution trace of TA is semantic equivalence to the execution sequence of FMU. Therefore, we can encode FMU into TA to analyse the behavior of FMU component without exploring its internal structure.

Given an FMU $F=(S,U,Y,D,s_{0},set,get,doStep)$, we encode the FMU into a timed automaton $A = (L,X,l_{0},E,E_{i},E_{o},I)$, the congruent relationship between them is as following:
\begin{itemize}
\item
$L$ is a set of finite states. Note that a location of $A$ is the abstraction of a state in $F$.
\item
The initial location of TA $l_{0}$ which $x:=0 \vert x \in X$ is such that $s$ is set to $s_{0}$ of $F$. 
\item
Each input variable $u \in U$ ranges over $E_{i} \cup \{absent\}$.
\item
Each output variable $y \in Y$ ranges over $E_{o} \cup \{absent\}$.
\item
An input event in $e \in E_{i}$ is such that the function $set$ of $F$ sets the input variable $u$ to a given value. 
\item
An output event in $e \in E_{o}$ indicates that the function $get$ of $F$ gets the output variable $y$. The set of values in the $E_{i}$ can be seen as $Y$ of $F$.  
\item
The communication between the network of TA is the same as the I/O dependencies information in FMU. $(u,y) \in D$ denotes that output $y$ depend on input $u$. The output events also depend on the input events in TA.
\item
For any $e \in E$ of A, there is a transition $s \xrightarrow{e} s^{\prime}$, which may be found after the function $doStep$ is executing. For instance, if there is a transition $l \xrightarrow{e} l^{\prime}$ in $A$, at the same time $doStep(s,h)$ may be called which indicates that $F$ accepts the time step $h$ and reaches the new state $s^{\prime}$. However, $F$ maybe rejects the time step, if there is a rollback behavior happens, the transition in TA could be a edge $l^{\prime} \xrightarrow{e} l$, which denotes that a location travels to the former location.

\end{itemize}
\begin{figure}[htbp]
	\centering	{\includegraphics[width=3.5in,height=2.5in]{fig/abstractRole.png}}
	\caption{Encoding rules from FMU to TA.}
	\label{fmutota}
\end{figure}

It is not easy to translate FMU to TA directly, we propose some encoding rules from FMU to TA. As we can see in the Fig.\ref{fmutota}, given a state $s_{i}$ at $t_{1}$ in FMU, the operation $Dostep$ makes FMU reach a new state $s_{i+1}$ at $t_{1}+step$. This situation can be encoded into a transition in TA, in which a location $L_{i}$ delays $step$ time and goes to a new location $L_{i}^{\prime}$.

For the operation $Rollback$, given a state $s{i}$ at $t_{1}$ in FMU, the FMU will do a step1 to $s_{i+1}$ at $t_{1}+step1$, and then, the operation $rollback$ makes FMU reach the former state $s_{i}$. For this situation, it can be encoded as: location $L_{i}$ delays step1 time and reach a new location $L_{i}^{\prime}$ after a transition, next returns to the former Location $L_{i}$. 

For the operation $prediction$, given a state $s_{i}$, FMU can get max step size ($step$) for next step, and then reach a new state $s_{i+1}$ at $t_{1}+step$. For TA, it gets max step size in location $L_{i}$, then it delays $step$ time and reach a new location $L_{i}^{\prime}$ .

For data exchange between two FMUs in state $s_{i}$ at $t_{1}$, they exchange data at $t_{1}$ and then do the same step to $s_{i+1}$. In TA, there will be a signal $io$ to make the two FMUs do the same step from $L_{i}$ to $L_{i+1}$ after data exchange.

Although there are semantic gaps between FMUs and timed automata, we provide appropriate encoding rules to formalism FMU with timed automata. It lays the foundation for analyse FMI co-simulation with timed automata-based model checking.



\section{Modeling and Analysis of Master Algorithm}
\label{sec:ma}
The MA provides the orchestration of FMUs, which denotes the co-simulation of various FMUs. To ensure the correctness of coordination, it is necessary to verify certain properties of the MA. In this section, we model three versions of MAs with TA and verify some expected properties of MAs such as deadlock, livelock and reachability with UPPAAL.
\subsection{I/O Dependency Information}
When it comes to co-simulation, I/O dependency information \cite{BromanBGLMTW13} is inevitably required to be well considered. The MA calls function $Set$ to provide input value to an FMU and function $Get$ to obtain an output value. It is of vital importance to know the dependence between input and output of FMUs. The direct dependency information can be used to call the function $Set$ and $Get$ in a well-defined order, when you design the MA. In FMI 2.0, this information can be provided using the element $ModelStructure$ \cite{FMI2INTRO}. However, sometime there may be an algebraic loop generated by the sequence of function call, which may not converge. In section \ref{sec:sysml}, we presents water tank system to show how to detect algebraic loops in the architecture.
\subsection{Master Algorithm}
The MA is used to orchestrate the execution of different components or subsystems. Each subsystem serves as an FMU component whose simulation is triggered by a particular MA. FMUs can be seen as black boxes which can be simulated independently until it needs to exchange data or synchronize. There are three versions of MAs, which are shown in Fig.~\ref{ad-fixedstep}.
\begin{figure}[htbp]
\centering{
		\subfigure[Fix step size algorithm]{\includegraphics[width=3.5in,height=0.8in]{fig/MA1.png}
			\label{sd_fixedstep}}
		\hfil
		\subfigure[Rollback algorithm]{\includegraphics[width=3.5in,height=0.9in]{fig/MA2.png}
			\label{sd-rollback}}	
		\subfigure[Predictable step sizes algorithm]{\includegraphics[width=3.5in,height=1.1in]{fig/MA3.png}
			\label{sd-pre}}		
	\caption{Activity diagrams for three versions of MAs.}
	\label{ad-fixedstep}
	}
\end{figure}
\subsubsection{Fixed Step Algorithm}
For fixed step algorithm, all FMUs have the same step size. When MA calls $doStep$ with the step size $h$, it will advance from a communication point $t$ to the next communication point $t+h$. During the simulation step, an FMU with its own solver will simulate independently according to its input value and generate a running result as output value. MA will wait until all FMUs finish their simulation step and then get their output values to exchange data for preparing the next simulation step. The activity diagram for fixed step algorithm is illustrated in Fig.\ref{sd_fixedstep}. There are mainly three activities in the control flow: $initialize$, $doStep$ and $continue$. In the fixed step algorithm \cite{BromanBGLMTW13}, the co-simulation process should be reliable, when all FMUs are reliable. When some error happens during a simulation step, the co-simulation will be affected due to the wrong simulation step. To overcome the shortcoming of the fixed step algorithm, it needs rollback mechanism.
\subsubsection{Rollback Algorithm}
There are some important features proposed in the FMI 2.0. It supports to save the FMU state if necessary and the saved state can be restored. For example, MA calls $doStep$ on $FMU_{1}$ and $FMU_{2}$, $FMU_{1}$ accepts the request and $FMU_{2}$ rejects it. If we save the state of $FMU_{1}$ and $FMU_{2}$ at the communication point, we can restore the scene after $FMU_{2}$ rejects $doStep$. The activity diagram of rollback algorithm \cite{BromanBGLMTW13} is clearly shown in Fig.\ref{sd-rollback}. Compared with the fixed step algorithm, all FMUs are required to support $rollback$ mechanism, that is, all FMUs could return to the previous state if the step sizes of all FMUs simulation are not equal.
\subsubsection{Predictable Step Size Algorithm}
To improve the efficiency of MA, it is important to predict the step size. So, predictable step size algorithm was proposed \cite{BromanBGLMTW13}. The function $GetMaxStepSize$ was introduced to optimize the performance of rollback algorithm. This function returns the maximum step size and state flag of a predictable FMU. Maximum step is the largest step that a predictable FMU can perform. State flag includes $ok$, $discard$ and $error$. $OK$ denotes the predictable FMU can accept the simulation step size. $Discard$ denotes the predictable FMU only implement partial step during simulation. $Error$ denotes the predictable FMU can't continue the simulation because of its unacceptable state or unreasonable input value. When $discard$ and $error$ occur, the FMU needs to rollback to the previous saved state. Whether an FMU is predictable or not, it should be indicated in FMU's $xml$ file. Moreover, if an FMU supports rollback and predictable step size at the same time, the predictable step size algorithm can get the maximum step size of the FMU using $GetMaxStepSize$ function.

In predictable step size algorithm, MA chooses the maximum step size of all predictable FMUs and finds the smallest communication step size $h$ which ensure all predictable step size can be accepted. Then, the states of all FMUs are saved. MA calls $doStep(h)$ function of FMUs which support rollback. The function $doStep()$ will return the real performed step size. If all performed step sizes are equal to $h$, MA will call $doStep(h)$ for FMUs. Otherwise, MA will find the smallest performed step $h_{min}$, then all FMUs will restore the saved state. Finally, MA will invoke $doStep$ $(h_{min})$ on all FMUs. The control flow of predictable step size algorithm is shown in Fig.\ref{sd-pre}. For example, $getMinStepFromCP$ is an activity that MA will call $GetMaxStepSize$ on all predictable FMUs to find their maximum simulation step size and then chooses the smallest one of them. 

\subsection{Modeling and Verification of MA} 
UPPAAL is a toolset for verification of real-time systems represented by a network of TA which is extended with integer variables, structured data types, and channel synchronization. We model the MAs using TA in UPPAAL. The Fig.\ref{ta-master} shows the TA models of three MAs,  respectively. Fixed step algorithm has $Init$, $doStep$ states and synchronize with $FMU$ by channel $continue$. Rollback algorithm has $Init$, $DoStep$, and $Continue$ states. If all FMUs don't have the same step size, rollback algorithm will communicate with FMUs by $rollback$ signal, otherwise, it will send $continue$ signal and move to $Continue$ state. Predictable step size algorithm has $Init$, $find \_ CP \_ MIN$, $DoStep$, $writeCP$ states. It obtains the minimal step size $step2$ of FMUs supporting $GetMaxStepSize$ function and the maximal step size $step1$ of FMUs supporting rollback. If $step1$ is greater than $step2$, FMUs receive $rollback$ signal and return to $DoStep$ state. Otherwise, FMUs receive $continue$ signal and do next step.  

\begin{figure}[htbp]
\centering{
		\subfigure[TA for fixed step algorithm]{\includegraphics[width=1.0in,height=0.8in]{fig/fixedstep_master.png}
			\label{ta_fixedstep}}
		\hfil
		\subfigure[TA for rollback algorithm]{\includegraphics[width=2.0in,height=1.2in]{fig/rollback_master.png}
			\label{ta-rollback}}	
		\subfigure[TA for predictable step size algorithm]{\includegraphics[width=3.5in,height=1.6in]{fig/pma_master.png}
			\label{ta-pre}}		
	\caption{TA for three versions of MAs.}
	\label{ta-master}
	}
\end{figure}

We verify the properties of three various MAs including reachability, liveness and deadlock. Experimental results are shown in Table \ref{ta_r}.

\begin{itemize}
\item
$E \langle\rangle ~master.dostep$, $E\langle\rangle~master.Continue$ and $E\langle\rangle~master.writeCP$ are reachability properties checking whether the system can reach these states;
\item
$master.Init \rightarrow master.dostep$, $master.Init \rightarrow master.Continue$ and $master.Init \rightarrow master.Continue$ are liveness properties. If the MA arrives at the former state, it eventually reaches the latter state;
\item
$A[]~not~deadlock$ is safety property, which means whether the MA will be deadlock.
\end{itemize}

Table \ref{ta_r} shows that the properties such as deadlock, liveness and reachability are satisfied, which ensures that the correctness of MA. For example, The property $A[]~not~deadlock$ is satisfied, which means the MA is deadlock free. The property $master.Init \rightarrow master.doStep$ is satisfied, which means if the model reach the former state $Init$, it will eventually reach the state $doStep$. The property $E\langle\rangle~master.doStep$ is satisfied, which means there exists a reachable state $doStep$.  In short, the coordination of CPS involves architecture and MA. In this section, we have verified the correctness of various MAs, which will be applied to the case study.
\begin{table}
\caption{Experimental results for verifying MA}
\centering
\begin{tabular}{c c c}
        \hline
        MA & Verified Property & Result\\
        \hline
        \multirow{2}{2.0cm}{Fixed Step}
                & $A[]~not~deadlock$ & True\\
                & $master.Init \rightarrow master.dostep$ & True\\
                & $E\langle\rangle~master.dostep$ & True\\

        \hline
        \multirow{2}{2.0cm}{Rollback}
                & $A[]~not~deadlock$ & True\\
                & $master.Init \rightarrow master.Continue$ & True\\
                & $E\langle\rangle~master.Continue$ & True\\

        \hline
        \multirow{2}{2.0cm}{Predictable}
                & $A[]~not~deadlock$ & True\\
                & $master.Init \rightarrow  master.writeCP$ & True\\
                & $E\langle\rangle~master.writeCP$ & True\\
        \hline
\end{tabular}
\label{ta_r}
\end{table}



\section{Case study}
\label{sec:sysml}
To illustrate our approach, we take an example (water tank) inspired by \cite{AmalioPCW16}. According to the I/O dependency information between FMUs, the architectural model for water tank is constructed using SysML. The aim of using SysML is to design the architecture of the system with a more high-level modeling language. It helps to show the components and their connection.

%\subsection{Case Study: Water Tank}
The water tank system is our running example. A source of water flows into the water tank whose water flows into the drain. The source is controlled by a valve; when the valve is open, the water flows into the water tank. The valve, managed by a software controller, is opened or closed stochastically or depending on the water level. There are three various water tank systems depending on various connections between controller, valve and tank. 

\subsection{Architecture Modelling in SysML}
SysML is a general purpose domain-specific language (DSL) \cite{SemerathBHSV17} for model-based systems engineering (MBSE) \cite{Dori16}, which is originated as an initiative of the International Council on Systems Engineering (INCOSE) \cite{Pepper2015International} in January 2001. SysML is implemented as a UML profile. The \textit{Block Definition Diagram}(BDD) describes the system blocks and their features (structural and behavioural). The\textit{Connection Diagram} (CD) describes the internal structure of blocks. The ports of blocks are connected by the connector. The I/O dependence of blocks describes the communication between blocks. SysML block diagrams are usually used to describe the architecture of systems.

Figure~\ref{myad} shows the block definition diagram for the water tank system. The system consists of three blocks, i.e., \emph{Valve}, \emph{Tank} and \emph{Controller}, in which \emph{Valve} and \emph{Tank} are physical components. \emph{Controller} is the cyber component. Each component has its own input and output. For instance, the input interface of \emph{Valve} is named as \emph{vin}, which is used to input the \emph{Open-Closed} signal. 
\begin{figure}[htbp]
	\centering	{\includegraphics[width=3.2in,height=2.3in]{fig/AD.jpg}}
	\caption{SysML BDD for water tank system.}
	\label{myad}
\end{figure}
Figure~\ref{cd} shows the connection diagram for the system. There are three cases for connections. The first case is that the system has one valve, one controller and one tank. The controller sends stochastic signals to control the valve on/off leading to various rate of water flow. The second case is that the signal from the controller is affected by the water level of the tank. The last case is on the basis of the first case and adds another tank2 which is affected by the flow rate of the tank1.

\begin{figure}[htbp]
\centering{
		\subfigure[Connection case 1]{\includegraphics[width=3.2in,height=0.8in]{fig/CD1.png}
			\label{cd1}}
		\hfil
		\subfigure[Connection case 2]{\includegraphics[width=3.2in,height=0.8in]{fig/CD2.png}
			\label{cd2}}
		\hfil
		\subfigure[Connection case 3]{\includegraphics[width=3.2in,height=0.8in]{fig/CD3.png}
			\label{cd3}}
	\caption{SysML CD for water tank system.}
	\label{cd}
	}
\end{figure}
We model the architecture with SysML which is a high-level modeling language. The SysML BDD shows the blocks of system and SysML CD shows the connection between blocks. In next section, we abstract each block as a FMU, and obtain the connection between FMUs based on the SysML CD.

\subsection{The FMUs  Connection of Water Tank System}
Figure~\ref{fmu-con} is the FMUs and FMUs connection of water tank system. There are three connection cases between the FMUs according to the SysML CD in the previous section. The first case contains three FMU components (\emph{Controller}, \emph{Valve} and \emph{ Tank1}) and two channels($v \_ vin$, $w \_ win$) as shown in Fig.\ref{fmu-con1}. The controller and valve are connected with channel $v \_ vin$. The valve and tank1 are connected with channel $w \_ win$. The second case is shown in Fig.\ref{fmu-con2}, there could be a channel $sout \_ s$ between tank1 and controller, which means the water level of tank1 affects the control strategy of the controller. Figure~\ref{fmu-con3} shows the third case, there could be another (tank2), the tank1 and tank2 are connected by the channel $w \_ out$. 
\begin{figure}[htbp]
\centering{
		\subfigure[Connection case 1]{\includegraphics[width=1.0in,height=0.6in]{fig/fmuc1.png}
			\label{fmu-con1}}
		\hfil
		\subfigure[Connection case 2]{\includegraphics[width=1.0in,height=0.6in]{fig/fmuc2.png}
			\label{fmu-con2}}
		\hfil
		\subfigure[Connection case 3]{\includegraphics[width=1.0in,height=0.6in]{fig/fmuc3.png}
			\label{fmu-con3}}
	\caption{FMUs connection of water tank system.}
	\label{fmu-con}
	}
\end{figure}
How can we assure the correctness of the architecture models? We attempt to verify it with model checking based on timed automata. More details on verification process can be found in the next section.
\subsection{Verification and Analysis with UPPAAL}
\label{sec:mauppaal}
This section performs a formal analysis of the architectures of water tank. First of all, we encode FMUs of the water tank and model the MA with TA which composes a network of TA. Next, the models are verified with the model checker UPPAAL. The execution of FMU and co-simulation is time-related. We abstract the execution of FMUs for the water tank and encode it with the locations and transitions of TA according to the encoding rules proposed in Section \ref{sec:encoding}. Besides, we also model the MA as a TA to coordinate the execution between several FMUs. The TA templates for FMUs and MA are shown in Fig.\ref{tk-arch1}. Here, we adopt the rollback MA to coordinate the FMUs. The other two MAs can be analysed with the similar way. We do not present the details of them due to the space limits of this paper.

\begin{figure}[htbp]
\centering{
		\subfigure[TA for FMU\_controller]{\includegraphics[width=1.6in,height=1.0in]{fig/2signal_controller.png}
			\label{tk_controller}}
		\hfil
		\subfigure[TA for FMU\_valve]{\includegraphics[width=1.6in,height=1.0in]{fig/2signal_v.png}
			\label{tk_v}}
			
	    \subfigure[TA for FMU\_WaterTank1]{\includegraphics[width=1.5in,height=1.2in]{fig/2signal_wt1.png}
			\label{tk_wt1}}
		\hfil
		 \subfigure[TA for rollback MA]{\includegraphics[width=1.7in,height=1.2in]{fig/2signal_master.png}
			\label{tk_ma}}		
	\caption{TA Network for connection case 1: $controller$ $\vert\vert$ $valve$ $\vert\vert$ $WaterTank1$ $\vert\vert$ $MA$.}
	\label{tk-arch1}
	}
\end{figure}

\begin{figure}[htbp]
\centering{
		\subfigure[Execution trace]{\includegraphics[width=1.6in,height=1.8in]{fig/trs.png}
			\label{trs}}
		\hfil
		\subfigure[Execution sequence diagram]{\includegraphics[width=1.6in,height=1.8in]{fig/seq.png}
			\label{seq}}
	\caption{The execution fragment of the coordination in UPPAAL.}
	\label{trs-seq}
	}
\end{figure}

Fig.~\ref{tk_controller}, \ref{tk_v}, \ref{tk_wt1} are the templates for $controller$, $valve$ and $WaterTank1$ respectively, which model FMUs of the water tank. These FMUs have four key states: $start$, $dostep$, $Rollback$ and $reset$. Fig.~\ref{tk_controller} shows the template for $controller$ which executes with random step size. It synchronizes with $valve$ by signal $v$ and transfers to $Rollback$ state, and then waits for a signal from the MA. Until the $controller$ receives the $continue$ signal, it does data exchange with other FMUs, and returns to $start$ state. Otherwise, it receives $rollback$ signal, when it obtains the minimize step size of all FMUs, it transfers to $Rollback$ state. The states and transitions of $valve$ and $WaterTank1$ template are similar with the template of $controller$. Fig.~\ref{tk_ma} shows the template for the MA. Firstly, the MA initializes the parameters, and then it gets minimize step size of FMUs until all FMUs visit $dostep$. Next, the MA decides which signal should be sent according to the guard. If the step sizes of all FMUs are equal, the MA will send $continue$ signal, otherwise, send $rollback$ signal.

Fig.~\ref{trs-seq} is the execution fragment of the coordination in UPPAAL, we can find that $valve$ sends a $w$ signal to perform data exchange with $WaterTank1$. After that, $WaterTank1$ moves to $dostep$ state. The MA broadcasts a $rollback$ signal to all templates, which leads to all of them arrive at $reset$ state. Finally, the MA sends a $continue$ signal to all FMUs. All templates return to $start$ state, and then do the next step. The execution fragments show that our models are correct.

In order to compare the behavior of three connection cases of water tank system, we also model the other two connection cases in UPPAAL. For connection case 2, we add channel $s$ in the templates for $controller$ and $WaterTank1$ as shown in Fig.\ref{tk-arch2}. For connection case 3, we create template for $WaterTank2$ and channel $w2$ as shown in Fig.\ref{arc3}. The other models are the same as models of connection case1. Limited to the length of this paper, we only show the templates for $controller$ and $WaterTank1$ of connection case 2 and template for $WaterTank2$ of connection case 3. In the next subsection, we verify some properties of various connection cases to detect whether there is an algebraic loop in the architecture.
\begin{figure}[htbp]
\centering{
		\subfigure[TA for FMU\_controller]{\includegraphics[width=1.8in,height=1.2in]{fig/2signal_cycle_controller.png}
			\label{tk2_controller}}
		\hfil
		\subfigure[TA for FMU\_WaterTank1]{\includegraphics[width=1.5in,height=1.2in]{fig/2signal_cycle_wt1.png}
			\label{tk2_v}}		
	\caption{TA for connection case 2.}
	\label{tk-arch2}
	}
\end{figure}
\begin{figure}[htbp]
	\centering	{\includegraphics[width=3.5in,height=1.2in]{fig/4signal_wt2.png}}
	\caption{TA for FMU\_WaterTank2 of connection case 3.}\label{arc3}
\end{figure}

UPPAAL supports a simplified version of TCTL \cite{BouchenebGR09} to specify the property. We verify the following properties of each connection case:
\begin{itemize}
\item
$E\langle\rangle~WT1.Rollback$ and $E\langle\rangle~master.Continue$ specify reachability properties. It means the FMU of $WaterTank1$ will $Rollback$ and the MA will reach $Continue$ state.
\item
$master.start \rightarrow master.Continue$ specifies liveness property. It means once the MA start, it will continue eventually.
\item 
$A[]~not~deadlock$ specifies safety property. It means the execution of the system will not be deadlock.
\end{itemize}

The verification results are listed in Table \ref{rs}. We can find that all properties of connection case 1 and case 3 are satisfied. It shows that our MA works well and the composition of FMUs is determinate. However, the liveness and reachability properties of connection case 2 are not satisfied. It means there is an algebraic loop which may be introduced with the I/O dependency in the architecture. The experimental results show that our approach is feasible and useful for model checking the coordination of CPS. Here, we only focus on the detection of algebraic loop and the correctness of coordination. In the future work, we will consider how to eliminate the algebraic loop.  
\begin{table}
\caption{Experimental results for various connection case}
\centering
\begin{tabular}{c c c} 
        \hline  
        Connection case & Property & Result\\
        \hline
        \multirow{2}{2.0cm}{Case 1}  
                & $E\langle\rangle~WT1.Rollback$ & True\\ 
                & $E\langle\rangle~master.Continue$ & True\\ 
                & $master.start\rightarrow master.Continue$ & True\\ 
                & $A[]~not~deadlock$ & True\\   
        \hline 
        \multirow{2}{2.0cm}{Case 2}  
                & $E\langle\rangle~WT1.Rollback$ & True\\ 
                & $E\langle\rangle~master.Continue$ & False\\ 
                & $master.start\rightarrow master.Continue$ & False\\ 
                & $A[]~not~deadlock$ & True\\   
        \hline 
        \multirow{2}{2.0cm}{Case 3}  
                & $E\langle\rangle~WT1.Rollback$ & True\\ 
                & $E\langle\rangle~master.Continue$ & True\\ 
                & $master.start \rightarrow master.Continue$ & True\\ 
                & $A[]~not~deadlock$ & True\\   
        \hline 
\end{tabular} 
\label{rs}
\end{table}





\section{Related Works}
Statistical Model Checking technique was first proposed by R.Grosu \cite{grosu2005monte}. Some variations \cite{legay2010statistical} \cite{Younes2004Planning} \cite{younes2006statistical}\cite{jha2009bayesian} \cite{zuliani2013bayesian} \cite{herault2004} based on the basic SMC have been proposed in the past few years. Some related work are summarized as follows:

\textbf{Basic SMC.}
SMC refers to a series of simulation-based techniques that can be used to answer two questions: (1)Qualitative: Is the probability of model \emph{s} satisfying property $\phi$ greater than or equal to a certain threshold? and (2)Quantitative: What is the probability of model \emph{s} satisfying property $\phi$? For qualitative SMC, Kim.G.larsn et al. \cite{kim2012statistical} have given an empirical evaluation. BHT and SPRT are more effective than SSP. BHT generates more traces when checking the property whose estimation probability is close to its real probability, so SPRT is faster than BHT for this situation. For other situation, BHT is obviously more efficient than SPRT. For quantitative SMC, Zuliani et al. \cite{zuliani2013bayesian} have compared the number of traces analyzed by APMC and BIE, and they have concluded that BIE excels remarkably in performance. Our approach focuses on the performance of BIE algorithm.

\textbf{SMC with abstraction and learning.}
BIE algorithm needs more traces when checking the property whose probability is close to 0.5, while the number of traces is drastically reduced when the probability approaches to 0 or 1 \cite{zuliani2013bayesian}. In our recent work \cite{jiangkaiqiang2016}, we have partitioned the original probability space $\Omega$ into many sub-spaces $\Omega_1$,... ,$\Omega_m$, and evaluated the probability of each sub-space in parallel. Therefore, the trace number for evaluating the original probability will be decreased and depends on the maximum number of traces for evaluating sub-spaces theoretically. We find that the number of traces is effectively reduced while ensuring the accuracy of the probability within an acceptable error bound.

\textbf{Distributed SMC.}
As observed in \cite{younes2005ymer}, SMC algorithms can be distributed with master/slave architecture where multiple slave processes are used to generate traces. When working with an estimation algorithm, the number of traces for verifying the property is known in advance and can be equally distributed between the slaves. When working with the sequential algorithms, the situation gets more complicated, so we need to avoid introducing bias when collecting the traces generated by the slave processes. To solve this problem, H. L. S. Younes proposed a method in \cite{Younes2004Planning} where the bias is avoided by committing ,$\alpha$ $priori$, to the order in which observations will be taken into account. Peter Bulychev et al. generalized the above method with batches and buffer \cite{Bulychev2012Checking}. Batches aggregate the outcomes for reducing communication and the buffer is used to improve concurrency since the nodes are more loosely synchronized. They also implemented the distributed Hypothesis testing algorithm without introducing bias. The algorithm effectively reduce the time consumption for generating a single trace. Our work is different from the existing work, we use abstraction and learning technique (AL-SMC) to reduce the number of simulation traces, and adopt distributed technology with AL-SMC to reduce both the number of traces and time consumption for generating a single traces.   
\section{Conclusion and Future Work}
\label{sec:conclusion&ack}
This paper has presented our novel approach to check the FMI co-simulation , which facilitates the formal analysis of CPSs. This involves model checking the reachability, livelock and deadlock of three various master algorithms. Besides, the correctness and  relevant system properties of the architecture are also analysed. To achieve the goal, we encode the FMU and master algorithms with timed automata. Then the properties of the co-simulation are verified with UPPAAL. We evaluate this approach using the example water tank. The results show that our approach is feasible and useful.

An interesting direction of future work is that we attempt to analyse and compare the performance of various master algorithms in the future. Besides, more complex case studies will be conducted to check the scalability of proposed approach. The tool support for our approach should be improved further.
\section*{Acknowledgement}
This work was supported by NSFC (Grant No.61472140, 61202104) and NSF of Shanghai (Grant No. 14ZR1412500).





\bibliographystyle{splncs03}

\bibliography{refcoma}
\end{document}
